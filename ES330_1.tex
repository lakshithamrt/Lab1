\documentclass[11pt]{article}
\usepackage[margin=1.0in]{geometry}
\usepackage{amsmath}
%Gummi|063|=)
\title{\textbf{Homework No. 02}}
\author{Due on 5$^{th}$ of February}
\date{}
\usepackage{graphicx}
\begin{document}

\maketitle

\begin{enumerate}
\item The gage pressure of the air in the tank shown in figure 1 is measured to be 65 kPa. Determine the differential height {\it h} of the mercury column.({\bf Problem 3-49 in the text})
\begin{figure}
\centering
\includegraphics[scale=0.3]{fig1.jpg}
\caption{Figure for the problem no.1}
\label{fig1}
\end{figure}
 
\item The 500-kg load on the hydraulic lift shown in figure 2 is to be raised by pouring oil ( $\rho$=780 kg/m$^3$) in to a thin tube. Determine how high {\it h} should be in order to begin to raise the weight.({\bf Problem 3-51 in the text})
\begin{figure}
\centering
\includegraphics[scale=0.3]{fig2.jpg}
\caption{Figure for the problem no.2}
\label{fig1}
\end{figure}

\item The pressure difference between an oil pipe and water pipe is measured by a double fluid manometer, as shown in figure 3. For the given fluid heights and specific gravities, calculate the pressure difference {\bf $\Delta P= P_B-P_A$}.({\bf Problem 3-56 in the text})

\begin{figure}
\centering
\includegraphics[scale=0.3]{fig3.jpg}
\caption{Figure for the problem no.3}
\label{fig1}
\end{figure}


\item A 6m -high, 5 m-wide rectangular plate blocks the end of a 5 m-deep fresh water channel, as shown in the figure 4. The plate is hinged about a horizontal axis along its upper edge through a point A and is restrained from opening by a fixed ridge at point B. Determine the force exerted on the plate by the ridge.({\bf Problem 3-75 in the text})

\begin{figure}
\centering
\includegraphics[scale=0.5]{fig4.jpg}
\caption{Figure for the problem no.4}
\label{fig1}
\end{figure}



\end{enumerate}


 
\end{document}

